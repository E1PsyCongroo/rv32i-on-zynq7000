\section{Checkpoint 4 - Optimization}

Checkpoint 4 is an optimization checkpoint lumped with the final checkoff.
This part of the project is designed to give students freedom to implement the optimizations of their choosing to improve the performance of their processor.

The optimization goal for this project is to minimize the \textbf{execution time} on the \verb|mmult| program, as defined by the 'Iron Law' of Processor Performance.

\begin{equation*}
\frac{\text{Time}}{\text{Program}} = \frac{\text{Instructions}}{\text{Program}} \times \frac{\text{Cycles}}{\text{Instruction}} \times \frac{\text{Time}}{\text{Cycle}}
\end{equation*}

The number of instructions is fixed, but you have freedom to change the CPI and the CPU clock frequency.
Often you will find that you will have to sacrifice CPI to achieve a higher clock frequency, but there also will exist opportunities to improve one or both of the variables without compromises.

\subsection{Grading on Optimization: Frequency vs. CPI}
You must demonstrate that your processor has a working BIOS and can load and execute \textbf{mmult}.

Full credit will be awarded if you're able to evaluate different design trade-off points (at least three) between frequency and CPI of \textbf{mmult} (especially if you have implemented some interesting optimization for CPI and increase the frequency further would degrade the performance instead of helping).

When exploring design trade-offs, also consider the total FPGA resource utilization. You should note the number of LUTs, Block RAMs, FFs, and DSP Blocks that you use for each design point. You can find these numbers in \verb|hardware/build/impl/post_place_utilization.rpt|.

Also note that your final optimized design does not need to be strictly three-stage pipeline. Extra credit will be awarded based on additional optimizations listed in the extra credit section, please check with a GSI ahead of time if you are expanding to include these. If you have other ideas please check with a GSI to see if it can be awarded extra credit.

\subsection{Clock Generation Info + Changing Clock Frequency}
Open up \verb|z1top.v|.
There's top level input called \verb|CLK_125MHZ_FPGA|.
It's a 125 MHz clock signal, which is used to derive the CPU clock.

Scrolling down, there's an instantiation of \verb|clocks| (\verb|clocks.v|), which is a wrapper module of PLL (phase locked loop) primitives on the FPGA. This is a circuit that can create a new clock from an existing clock with a user-specified multiply-divide ratio.

The \verb|CLKIN1| (line 47) input clock of the PLL is driven by the 125 MHz \verb|CLK_125MHZ_FPGA|.
The frequency of \verb|CLKOUT0| (line 39) is calculated as:
\begin{equation*}
  \mathtt{CLKOUT0}\_f = \mathtt{CLKIN1}\_f \times \frac{\mathtt{CPU\_CLK\_CLKFPOUT\_MULT}}{\mathtt{CPU\_CLK\_DIVCLK\_DIVIDE} \times \mathtt{CPU\_CLK\_CLKOUT\_DIVIDE}}
\end{equation*}

Based on the default parameter values, we get the following CPU clock frequency:
\begin{equation*}
  \mathtt{CLKOUT0}\_f = 125 \text{ MHz} \times \frac{34}{5 \times 17} = 50 \text{ MHz}
\end{equation*}

To change the target CPU clock frequency, you must do \textbf{both} of the following:

\begin{enumerate}
\item Change the parameters (\verb|CPU_CLK_CLKFBOUT_MULT|, \verb|CPU_CLK_DIVCLK_DIVIDE|, 

\verb|CPU_CLK_CLKOUT_DIVIDE|) in z1top according to the table below

\item Change the \verb|CPU_CLOCK_FREQ| parameter in z1top to match the PLL parameters

\end{enumerate}

\begin{table}[hbt]
  \begin{center}
    \caption{PLL Settings}
    \label{mem_map1}
    \begin{adjustbox}{width=\columnwidth,center}
    \begin{tabular}{l l l l}
      \toprule
      \textbf{Frequency} & \textbf{DIVCLK\_DIVIDE} & \textbf{CLKFBOUT\_MULT} & \textbf{CLKOUT\_DIVIDE}\\
      \midrule
      50 MHz & 5 & 34 & 17 \\
      60 MHz & 5 & 36 & 15 \\
      65 MHz & 5 & 39 & 15 \\
      70 MHz & 5 & 42 & 15 \\
      75 MHz & 5 & 33 & 11 \\
      80 MHz & 5 & 48 & 15 \\
      85 MHz & 5 & 34 & 10 \\
      90 MHz & 5 & 36 & 10 \\
      95 MHz & 5 & 38 & 10 \\
      100 MHz & 5	& 36 & 9 \\   
      \bottomrule
    \end{tabular}
    \end{adjustbox}
  \end{center}
\end{table}

\subsection{Critical Path Identification}
After running \verb|make impl|, timing analysis will be performed to determine the critical path(s) of your design.
The timing tools will automatically figure out the CPU's clock timing constraint based on the PLL parameters you set in \verb|z1top.v|.

The critical path can be found by looking in

\verb|hardware/build/impl/post_route_timing_summary.rpt|.

Look for the paths within your CPU. There are 2 types of paths: max delay paths (setup) and min delay paths (hold).

For each timing path look for the attribute called ``slack''.
Slack describes how much extra time the combinational delay of the path has before/after the rising edge of the receiving clock.

For a \textbf{max delay path}, the slack is a \textbf{setup} time attribute.
Positive slack means that this timing path resolves and settles \textbf{before} the rising edge of the clock, and negative slack indicates a setup time violation.

For a \textbf{min delay path}, the slack is a \textbf{hold} time attribute.
Positive slack means that this timing path resolves and settles \textbf{after} the rising edge of the clock, and negative slack indicates a hold time violation.

There are some common delay types that you will encounter.
\verb|LUT| delays are combinational delays through a LUT.
\verb|net| delays are from wiring delays. They come with a fanout attribute which you should aim to minimize.
Notice that your logic paths are usually dominated by routing delay; as you optimize, you should reach the point where the routing and LUT delays are about equal portions of the total path delay.

\subsubsection{Schematic View}
To visualize the path, you can open Vivado (\verb|make vivado|), and open a DCP (Design Checkpoint) file (File → Checkpoint → Open). The DCP is in \verb|build/impl/z1top_routed.dcp|.

Re-run timing analysis with Reports → Timing → Report Timing Summary. Use the default options and click OK. Navigate (on the bottom left) to Intra-Clock Paths → \verb|cpu_clk| → Setup (or Hold).

You can double-click any path to see the logic elements along it, or you can right-click and select Schematic to see a schematic view of the path.

The paths in post-PAR timing report may be hard to decipher since Vivado does some optimization to move/merge registers and logic across module boundaries. It may help to look at the post-synth DCP in \verb|build/synth/z1top.dcp|. You can also use the \href{https://www.xilinx.com/support/answers/54778.html}{\texttt{keep\_hierarchy} attribute} to prevent Vivado from moving registers and logic across module boundaries (although this may degrade QoR).

\begin{minted}{verilog}
// in z1top.v
(* keep_hierarchy="yes" *) cpu #( ) cpu ( );
\end{minted}

\subsubsection{Finding Actual Critical Paths}
When you first check the timing report with a 50 MHz clock, you might not see your 'actual' critical path.
50 MHz is easy to meet and the tools will only attempt to optimize routing until timing is met, and will then stop.

You should increase the clock frequency slowly and rerun \verb|make impl| until you fail to meet timing.
At this point, the critical paths you see in the report are the 'actual' ones you need to work on.

Don't try to increase the clock speed up all the way to 100 MHz initially, since that will cause the routing tool to give up even before it tried anything.

\subsection{Optimization Tips}
As you optimize your design, you will want to try running \verb|mmult| on your newly optimized designs as you go along. You don't want to make a lot of changes to your processor, get a better clock speed, and then find out you broke something along the way.

You will find that sacrificing CPI for a better clock speed is a good bet to make in some cases, but will worsen performance in others.
You should keep a record of all the different optimizations you tried and the effect they had on CPI and minimum clock period; this will be useful for the final report when you have to justify your optimization and architecture decisions.

There is no limit to what you can do in this section.
The only restriction is that you have to run the original, unmodified \verb|mmult| program so that the number of instructions remain fixed.
You can add as many pipeline stages as you want, stall as much or as little as desired, or perform any other optimizations.
If you decide to do a more advanced optimization (like a 5 stage pipeline), ask the staff to see if you can use it as extra credit in addition to the optimization.

Keep notes of your architecture modifications in the process of optimization.
Consider, but don't obsess, over area usage when optimizing (keep records though).

\subsection{Checkoff}
Refer to \textbf{4.1}. You will run your new implementation on the FPGA again and will be graded based on the best \verb|mmult| performance you were able to achieve, but \textit{more critically} on how many design points you explored.



\subsection{Extra Credit}
\label{extra_credit}
Teams that have completed the base set of requirements are eligible to receive extra credit worth up to 10\% of the project grade by adding extra functionality and demonstrating it at the time of the final checkoff.

The following are suggested projects that may or may not be feasible in one week.
\begin{itemize}
  \item Improve Branch Predictor: Beyond our 2-bit saturating based BHT, you can improve the Branch Predictor from Checkpoint 3. You can come up with improved scheme, such as incorporating making the cache set associative (required for 251A students), incorporating global history, adding a Branch Target Buffer, etc. Whatever you choose to do, you must improve your CPI from checkpoint 3 to qualify for extra credit.
  \item 5-Stage Pipeline: Add more pipeline stages and push the clock frequency past 100MHz. Note that on some deeper pipelines, the \texttt{host\_to\_fpga} task does not wait enough clock cycles for the CPU to process a character before sending the next character over uart. You may want to bump up the wait cycles, e.g., 500 cycles, to avoid this (though it does slow down the testbench).
  \item RISC-V M Extension: Extend the processor with a hardware multiplier and divider
  \item Everything 100MHz or beyond: Push the frequency of the full \verb|z1top| to 100MHz or better.
\end{itemize}

When the time is right, if you are interested in implementing any of these, see the staff for more details.



\newpage
